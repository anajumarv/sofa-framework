\section{Collision Models}
\subsection{Ray Traced Collision Detection}
This module implements the algorithm described in the paper entitled "Ray-traced collision detection for deformable bodies" by E.Hermann F.Faure and B.Raffin. When two objects are in collision, a ray is shot from each surface vertex in the direction of the inward normal. A collision is detected when the first intersection belongs to an inward surface triangle of another body.  A contact force between the vertex and the matching point is then created. Experiments  show that this approach is fast and more robust than traditional proximity-based collisions.

 To speedup the  searching of elements that cross the ray,   we  stored all the triangles of each colliding objects in an  octree. Therefore we can easily navigate inside this octree and efficiently find the points crossing the ray. The octree structure allow us to have a satisfying performance independently from the size of the triangles used, which is not the case for  a regular grid.


\subsubsection{Using this module}
An exemple showing the usage of the Ray Traced collision detection can be found in the  RayTraceCollision.scn file in the \textit{scene} directory. The collision detection mechanism must be set as \textbf{RayTraceDetection}, and instead of using a TriangleModel one must use a \textbf{TriangleOctreeModel}. The TriangleOctreeModel will create an Octree that contains all the Triangles from the collision model. 
	
