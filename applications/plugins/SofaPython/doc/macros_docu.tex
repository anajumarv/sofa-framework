% ---- graphiques
\usepackage[pdftex]{graphicx}
\usepackage{wrapfig}
\usepackage{color}
\usepackage{pst-tree}
%\usepackage{hyperref}

% for accents
\usepackage[latin1]{inputenc}
\usepackage[T1]{fontenc}

\usepackage{algorithm}
\usepackage{algorithmic}

\definecolor{darkgreen}{rgb}{0,0.4,0}
\definecolor{darkblue}{rgb}{0,0,0.4}
\definecolor{darkgray}{rgb}{0.2,0.2,0.2}

% ---- inclusion de codes
\usepackage{listings}
\lstset{showstringspaces=false,tabsize=4,basicstyle=\scriptsize\sffamily,breaklines=true,breakatwhitespace=true,framexleftmargin=5mm, frame=shadowbox, framesep=1pt,rulesepcolor=\color{darkgray},rulesep=.5pt,keywordstyle=\bf\color{blue},commentstyle=\color{magenta},stringstyle=\color{red},numbers=left,numberstyle=\tiny,numbersep=5pt,columns=flexible}

\lstdefinestyle{bash}{language=bash}
\lstdefinestyle{Perl}{language=Perl}
\lstdefinestyle{Python}{language=Python}
\lstdefinestyle{C++}{language=C++,emph={__global__,__shared__,__syncthreads,blockIdx,threadIdx,float3,float4},emphstyle=\bf\color{darkgreen}}
\lstdefinestyle{DTD}{language=XML}
\lstdefinestyle{XML}{language=XML,usekeywordsintag=false,markfirstintag=true}
%begin{latexonly}
\newcommand{\includecode}[2]{
\lstinputlisting[style=#1]{#2}
}
%end{latexonly}


%\lstnewenvironment{code}{}{}
\lstnewenvironment{code_bash}{\lstset{style=bash}}{}
\lstnewenvironment{code_perl}{\lstset{style=Perl}}{}
\lstnewenvironment{code_python}{\lstset{style=Python}}{}
\lstnewenvironment{code_cpp}{\lstset{style=C++}}{}
\lstnewenvironment{code_dtd}{\lstset{style=DTD}}{}
\lstnewenvironment{code_xml}{\lstset{style=XML}}{}

\newcommand{\textcode}[1]{{\sf #1}}




\newcommand{\sofa}{SOFA}
\newcommand{\todo}[1]{}
\newcommand{\eg}{\textit{e.g.} }

\renewcommand{\vec}[1]{\ensuremath{\mathbf{#1 }}} % vector
\newcommand{\Vx}{\vec{x} } % position vector
\newcommand{\Vv}{\vec{v} } % velocity vector
\newcommand{\Va}{\vec{a} } % acceleration vector
\newcommand{\Vf}{\vec{f}} % force
\newcommand{\Vdv}{\vec{\delta\Vv}} % change of velocity vector (unknown in implicit CG, and used in constraint solver
\renewcommand{\P}{\mat{P} } % projection to a constrained space.

\newcommand{\JNL}{\mathbf{\mathcal{J}} }     % mapping des positions
\newcommand{\J}{\mat J }                 % mapping lineaire
\newcommand{\M}{\mat M }             % matrice de masse
\newcommand{\K}{\mat K }             % matrice de raideur
\newcommand{\B}{\mat B }             % matrice d'amortissement
\newcommand{\G}{\mat G }             % jacobien des contraintes



% ---- inclusion de codes
\definecolor{darkgreen}{rgb}{0,0.4,0}
\definecolor{darkblue}{rgb}{0,0,0.4}
\definecolor{darkgray}{rgb}{0.2,0.2,0.2}


% macros mathematiques
\newcommand{\ma}[1]{\ensuremath{\mathbf {#1}}}
\newcommand{\ve}[1]{\ensuremath{\mathbf {#1}}}

\usepackage{amsmath}
\usepackage{amsfonts}
\usepackage{amssymb}

% character styles
\newcommand{\bm}[1]{\ensuremath{\mathbf{{#1}}}}
\newcommand{\mcal}[1]{\mbox{$\mathcal #1$}} % rondes math
\newcommand{\bmcal}[1]{\mbox{\boldmath $\mathcal #1$}} % rondes grasses math
\newcommand{\ensemble}[1]{\mbox{$\mathbb{#1}$}}
\newcommand{\RRR}{\mbox{$\ensemble{R}^3$}} 


% d�finitions
\newcommand{\definition}[2]{\index{#1}{\bf #1}: #2}
\newcommand{\voc}[1]{\index{#1}#1}
\newcommand{\bvoc}[1]{\index{#1}{\bf #1}}

% misc
\newcommand{\EV}[1]{\stackrel{\rightarrow}{#1}}  % espace vectoriel
\newcommand{\EA}[1]{#1}                          % espace affine

% vectors, matrices
%\newcommand{\point}[1]{\mbox{$#1$}}          % un point
\newcommand{\point}[1]{\ensuremath{#1}}          % un point
\newcommand{\mat}[1]{\bm{#1}}         % matrice
\newcommand{\matnm}[3]{\bm{#1_{#2\times #3}}}  % matrice n lignes , m colonnes
\newcommand{\vect}[1]{\bm{#1}}        % vecteur 
%\newcommand{\vecf}[1]{\stackrel{\rightarrow}{#1}}  % vecteur avec fleche
\newcommand{\vecf}[1]{\mbox{$\overrightarrow{#1}$}}  % vecteur avec fleche
\newcommand{\ident}[1]{\bm{I_{#1}}}   % identit� en dimension n
\newcommand{\inv}[1]{#1^{-1}}         % matrice inverse
\newcommand{\psinv}[1]{#1^{+}}        % matrice pseudo-inverse
\newcommand{\transp}[1]{#1^T}         % transpos�e de 1
\newcommand{\trace}[1]{tr(#1)}        % trace
\newcommand{\deter}[1]{\mbox{$|#1|$}}       % determinant
\newcommand{\oppvec}[1]{\mbox{$\left( \vect {#1} \wedge \right)$}}  % operateur matriciel de produit vectoriel

% bases, reperes
\newcommand{\vecin}[2]{\mbox{${}^{#2}#1$}}    % vecteur 1 dans repere 2
\newcommand{\Base}[1]{\ensuremath{\mathcal B_{#1}}} % Symbole du repere 1
\newcommand{\chbase}[3]{\mbox{${}_{#2}^{#3}\mat{#1}$}}  % operateur 1 fait le passage de la base 3 vers la base 2
%\newcommand{\pchbase}[2]{\chbase{\mat{B}}{#1}{#2}}  % matrice de passage de la base 2 vers la base 1
\newcommand{\pchbase}[2]{\chbase{B}{#1}{#2}}  % matrice de passage de la base 2 vers la base 1
\newcommand{\Rep}[1]{\ensuremath{\mathcal R_{#1}}} % Symbole du repere 1
\newcommand{\rep}[1]{\Rep{#1}}                 % Symbole du repere 1
%\newcommand{\pchrep}[2]{\chbase{\mat{F}}{#1}{#2}}  % matrice de passage du repere 1 vers le repere 2, F comme Frame
\newcommand{\pchrep}[2]{\chbase{\bm{C}}{#1}{#2}}  % matrice de passage du repere 2 vers le repere 1

%% Operateur de passage du repere 1 par rapport a 2
%\newcommand{\ChgRep}[2]{\mbox{\boldmath $R_{#1}^{#2}$}}

% rotations	
%\newcommand{\rot}[2]{\mbox{$\mat{R}_{#1,#2}$}}      % rotation vectorielle
\newcommand{\rot}[2]{\ensuremath{\mat{R}_{#1,#2}}}      % rotation vectorielle
\newcommand{\rota}[3]{\mbox{$\mat{R}_{#1,#2,#3}$}}  % rotation affine

% translation
\newcommand{\trans}[2]{\mbox{$\chbase{\vect{t}}{#1}{#2}$}} % passage de #1 vers #2 par une translation, ou translation du repere #2 par rapport au repere #1

% vitesses et acc�l�rations
\newcommand{\VRep}[2]{\mbox{\boldmath $\dot R_{#1}^{#2}$}} % vitesse du repere 1 par rapport a 2 
%\newcommand{\Point}[2]{\mbox{\boldmath ${#1}^{#2}$}}  % Coordonnees d'un point 1 dans un repere 2
\newcommand{\Point}[2]{\mbox{$\vecin{\bm{#1}}{#2}$}}  % Coordonnees d'un point 1 dans un repere 2
\newcommand{\VPoint}[2]{\mbox{\boldmath ${\dot #1}_{/#2}$}} % Vitesse d'un point par rapport � un repere
\newcommand{\APoint}[2]{\mbox{\boldmath ${\ddot #1}_{/#2}$}} % Acceleration d'un point par rapport � un repere

% cinematique du solide
\newcommand{\derivedans}[2]{\mbox{$\dot{#1}^{(#2)}$}}  % derivee du vecteur 1 dans repere 2
\newcommand{\fixedans}[2]{\mbox{$#1_{\in #2}$}}        % vecteur 1 fixe dans repere 2
\newcommand{\vecom}{\mbox{$\bm{\Omega}$}}  % omega de 1 par rapport a 2
\newcommand{\vecrot}[2]{\mbox{$\vecom_{#1/#2}$}}  % omega de 1 par rapport a 2
\newcommand{\accrot}[2]{\mbox{$\dot{\vecom}_{#1/#2}$}}  % omega de 1 par rapport a 2
\newcommand{\vfdans}[3]{\mbox{$\vec V^{#2/#3}_{#1}$}}    % vitesse de 1 fixe dans 2 par rapport a 3
\newcommand{\afdans}[3]{\mbox{$\vec \Gamma^{#2/#3}_{#1}$}}    % acceleration de 1 fixe dans 2 par rapport a 3
\newcommand{\vmdans}[2]{\mbox{$\vec V^{/{#2}}_{#1}$}}    % vitesse de 1 mobile dans 2
\newcommand{\amdans}[2]{\mbox{$\vec \Gamma^{/#2}_{#1}$}}    % acceleration de 1 mobile dans 2

% chaines articulees
\newcommand{\liaison}[2]{\mbox{$\mathcal L_{#1,#2}$}}  % liaison du pere 1 vers fils 2 (et repere intermediaire)
\newcommand{\liaisonprime}[2]{\mbox{$\mathcal L'_{#1,#2}$}}  % deuxieme repere intermediaire de la liaison du pere 1 vers fils 2
\newcommand{\liaisonP}[2]{\mbox{$\mathcal L_{#1,#2}$}}  % Repere dans pere 1 de la liaison vers fils 2 
\newcommand{\liaisonC}[2]{\mbox{$\mathcal L'_{#1,#2}$}}  % Repere dans fils de la liaison du pere 1 vers fils 2 
%\newcommand{\transP}[2]{\pchrep{\liaisonP{#1}{#2}}{#1}}  % Matrice du repere dans pere de la liaison du pere 1 vers fils 2 
%\newcommand{\transC}[2]{\pchrep{\liaisonC{#1}{#2}}{#2}}  % Matrice du repere dans pere de la liaison du pere 1 vers fils 2 
%\newcommand{\transPC}[2]{\pchrep{\liaisonC{#1}{#2}}{\liaisonP{#1}{#2}}}  % matrice de passage entre repere liaison dans fils et repere de liaison dans pere
\newcommand{\transP}[2]{\chbase{C_p}{#2}{#1}}  % Matrice du repere dans pere de la liaison du pere 1 vers fils 2 
\newcommand{\transC}[2]{\chbase{C_c}{#2}{#1}}  % Matrice du repere dans pere de la liaison du pere 1 vers fils 2 
\newcommand{\transPC}[2]{\chbase{C_l}{#2}{#1}}  % matrice de passage entre repere liaison dans fils et repere de liaison dans pere
% \pchrep{fils}{pere} = \liaisonP{pere}{fils}\deplPC{pere}{fils}\liaisonC{pere}{fils}


\newcommand{\pctab  }{\hspace{0.15in}      }  % Pseudo-code indentation.
\newcommand{\code}[1]{ 
\begin{makeimage}
\begin{tabbing} \pctab \= \pctab \= \pctab \= \pctab \= \pctab \= \pctab \= \pctab \kill
#1
\end{tabbing}
\end{makeimage}
}
